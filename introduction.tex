
\section{Introduction}

\glspls{MSR} are difficult to model in fuel cycle
simulations due to their continuous reprocessing, where
there is a continuous flow of materials in and out of
the reactors. Many modules have been developed to model
continuous reprocessing in an \gls{MSR}. 



\subsection{\Cyclus}

\Cyclus is an agent-based fuel cycle simulation framework 
\cite{huff_fundamental_2016}, which means 
that each reactor, reprocessing plant, and fuel fabrication plant is modeled as an agent.
A \Cyclus simulation contains prototypes, which are fuel cycle facilities with
pre-defined parameters, that are deployed in the simulation as \texttt{facility} agents.
Encapsulating the \texttt{facility} agents are the \texttt{institution} and \texttt{region}.
A \texttt{region} agent holds a set of \texttt{institution}s.
An \texttt{institution} agent can deploy or decommission \texttt{facility} agents.
The \texttt{institution} agent is part of a \texttt{region} agent,
which can contain multiple \texttt{institution} agents. Several versions of \texttt{Institution}
and \texttt{region} exist, varying in complexity and functions \cite{huff_extensions_2014}.
 \texttt{DeployInst} is used as the institution archetype for this work, where the institution
deploys agents at user-defined timesteps.

At each timestep (one month),
agents make requests for materials or bid to supply them and exchange
with one another. A market-like mechanism called the dynamic resource exchange
\cite{gidden_agent-based_2015} governs the exchanges.
Each material resource has a quantity, composition, name, and a unique identifier
for output analysis. The timestep execution in \Cyclus follows 
\texttt{Build, Tick,} \gls{DRE}, \texttt{Tock, and Decommission}, as illustrated in
figure \ref{fig:time}. The \texttt{Tick}, and \texttt{Tock} phases are for
each agent to perform actions, such as transmutation, separation, or generation
of materials before and after the market exchange phase.

\begin{figure}[h]
\centering
\scalebox{0.7}{
\begin{tikzpicture}[node distance=1.5cm]
\node (Build) [process] {Build (kernel)};
\node (Tick) [process, below of=Build] {Tick (agent)};
\node (DRE) [process, below of=Tick]{Dynamic Resource Exchange (kernel) };
\node (Tock) [process, below of=DRE]{Tock (agent)};
\node (Decom) [process, below of=Tock] {Decommission (kernel)};

\draw [arrow] (Build) -- (Tick); 
\draw [arrow] (Tick) -- (DRE);
\draw [arrow] (DRE) -- (Tock);
\draw [arrow] (Tock) -- (Decom);
\end{tikzpicture}
}
\caption{\Cyclus timestep execution steps.}
\label{fig:time}
\end{figure}

The modularity of \Cyclus allows a low barrier of
entry for developers, since developers can create an
archetype (e.g. Reactor module, Reprocessing module)
without extensive knowledge of the \Cyclus framework.

\subsection{Saltproc}
Saltproc is an external python driver developed to
model batch-wise reprocessing in \glss{MSR}. The code
repeatedly runs SERPENT, performs material operations (e.g. reprocessing,
inflow of fertile material), and saves all material flow
and k-eff values in a HDF5 database. 
